\documentclass[]{article}
\usepackage{lmodern}
\usepackage{amssymb,amsmath}
\usepackage{ifxetex,ifluatex}
\usepackage{fixltx2e} % provides \textsubscript
\ifnum 0\ifxetex 1\fi\ifluatex 1\fi=0 % if pdftex
  \usepackage[T1]{fontenc}
  \usepackage[utf8]{inputenc}
\else % if luatex or xelatex
  \ifxetex
    \usepackage{mathspec}
  \else
    \usepackage{fontspec}
  \fi
  \defaultfontfeatures{Ligatures=TeX,Scale=MatchLowercase}
\fi
% use upquote if available, for straight quotes in verbatim environments
\IfFileExists{upquote.sty}{\usepackage{upquote}}{}
% use microtype if available
\IfFileExists{microtype.sty}{%
\usepackage{microtype}
\UseMicrotypeSet[protrusion]{basicmath} % disable protrusion for tt fonts
}{}
\usepackage[margin=1in]{geometry}
\usepackage{hyperref}
\hypersetup{unicode=true,
            pdftitle={Iris2},
            pdfauthor={Eric},
            pdfborder={0 0 0},
            breaklinks=true}
\urlstyle{same}  % don't use monospace font for urls
\usepackage{color}
\usepackage{fancyvrb}
\newcommand{\VerbBar}{|}
\newcommand{\VERB}{\Verb[commandchars=\\\{\}]}
\DefineVerbatimEnvironment{Highlighting}{Verbatim}{commandchars=\\\{\}}
% Add ',fontsize=\small' for more characters per line
\usepackage{framed}
\definecolor{shadecolor}{RGB}{248,248,248}
\newenvironment{Shaded}{\begin{snugshade}}{\end{snugshade}}
\newcommand{\KeywordTok}[1]{\textcolor[rgb]{0.13,0.29,0.53}{\textbf{#1}}}
\newcommand{\DataTypeTok}[1]{\textcolor[rgb]{0.13,0.29,0.53}{#1}}
\newcommand{\DecValTok}[1]{\textcolor[rgb]{0.00,0.00,0.81}{#1}}
\newcommand{\BaseNTok}[1]{\textcolor[rgb]{0.00,0.00,0.81}{#1}}
\newcommand{\FloatTok}[1]{\textcolor[rgb]{0.00,0.00,0.81}{#1}}
\newcommand{\ConstantTok}[1]{\textcolor[rgb]{0.00,0.00,0.00}{#1}}
\newcommand{\CharTok}[1]{\textcolor[rgb]{0.31,0.60,0.02}{#1}}
\newcommand{\SpecialCharTok}[1]{\textcolor[rgb]{0.00,0.00,0.00}{#1}}
\newcommand{\StringTok}[1]{\textcolor[rgb]{0.31,0.60,0.02}{#1}}
\newcommand{\VerbatimStringTok}[1]{\textcolor[rgb]{0.31,0.60,0.02}{#1}}
\newcommand{\SpecialStringTok}[1]{\textcolor[rgb]{0.31,0.60,0.02}{#1}}
\newcommand{\ImportTok}[1]{#1}
\newcommand{\CommentTok}[1]{\textcolor[rgb]{0.56,0.35,0.01}{\textit{#1}}}
\newcommand{\DocumentationTok}[1]{\textcolor[rgb]{0.56,0.35,0.01}{\textbf{\textit{#1}}}}
\newcommand{\AnnotationTok}[1]{\textcolor[rgb]{0.56,0.35,0.01}{\textbf{\textit{#1}}}}
\newcommand{\CommentVarTok}[1]{\textcolor[rgb]{0.56,0.35,0.01}{\textbf{\textit{#1}}}}
\newcommand{\OtherTok}[1]{\textcolor[rgb]{0.56,0.35,0.01}{#1}}
\newcommand{\FunctionTok}[1]{\textcolor[rgb]{0.00,0.00,0.00}{#1}}
\newcommand{\VariableTok}[1]{\textcolor[rgb]{0.00,0.00,0.00}{#1}}
\newcommand{\ControlFlowTok}[1]{\textcolor[rgb]{0.13,0.29,0.53}{\textbf{#1}}}
\newcommand{\OperatorTok}[1]{\textcolor[rgb]{0.81,0.36,0.00}{\textbf{#1}}}
\newcommand{\BuiltInTok}[1]{#1}
\newcommand{\ExtensionTok}[1]{#1}
\newcommand{\PreprocessorTok}[1]{\textcolor[rgb]{0.56,0.35,0.01}{\textit{#1}}}
\newcommand{\AttributeTok}[1]{\textcolor[rgb]{0.77,0.63,0.00}{#1}}
\newcommand{\RegionMarkerTok}[1]{#1}
\newcommand{\InformationTok}[1]{\textcolor[rgb]{0.56,0.35,0.01}{\textbf{\textit{#1}}}}
\newcommand{\WarningTok}[1]{\textcolor[rgb]{0.56,0.35,0.01}{\textbf{\textit{#1}}}}
\newcommand{\AlertTok}[1]{\textcolor[rgb]{0.94,0.16,0.16}{#1}}
\newcommand{\ErrorTok}[1]{\textcolor[rgb]{0.64,0.00,0.00}{\textbf{#1}}}
\newcommand{\NormalTok}[1]{#1}
\usepackage{graphicx,grffile}
\makeatletter
\def\maxwidth{\ifdim\Gin@nat@width>\linewidth\linewidth\else\Gin@nat@width\fi}
\def\maxheight{\ifdim\Gin@nat@height>\textheight\textheight\else\Gin@nat@height\fi}
\makeatother
% Scale images if necessary, so that they will not overflow the page
% margins by default, and it is still possible to overwrite the defaults
% using explicit options in \includegraphics[width, height, ...]{}
\setkeys{Gin}{width=\maxwidth,height=\maxheight,keepaspectratio}
\IfFileExists{parskip.sty}{%
\usepackage{parskip}
}{% else
\setlength{\parindent}{0pt}
\setlength{\parskip}{6pt plus 2pt minus 1pt}
}
\setlength{\emergencystretch}{3em}  % prevent overfull lines
\providecommand{\tightlist}{%
  \setlength{\itemsep}{0pt}\setlength{\parskip}{0pt}}
\setcounter{secnumdepth}{0}
% Redefines (sub)paragraphs to behave more like sections
\ifx\paragraph\undefined\else
\let\oldparagraph\paragraph
\renewcommand{\paragraph}[1]{\oldparagraph{#1}\mbox{}}
\fi
\ifx\subparagraph\undefined\else
\let\oldsubparagraph\subparagraph
\renewcommand{\subparagraph}[1]{\oldsubparagraph{#1}\mbox{}}
\fi

%%% Use protect on footnotes to avoid problems with footnotes in titles
\let\rmarkdownfootnote\footnote%
\def\footnote{\protect\rmarkdownfootnote}

%%% Change title format to be more compact
\usepackage{titling}

% Create subtitle command for use in maketitle
\newcommand{\subtitle}[1]{
  \posttitle{
    \begin{center}\large#1\end{center}
    }
}

\setlength{\droptitle}{-2em}

  \title{Iris2}
    \pretitle{\vspace{\droptitle}\centering\huge}
  \posttitle{\par}
    \author{Eric}
    \preauthor{\centering\large\emph}
  \postauthor{\par}
      \predate{\centering\large\emph}
  \postdate{\par}
    \date{1 février 2019}


\begin{document}
\maketitle

\section{Nous allons etudier 4 variables
aleatoire}\label{nous-allons-etudier-4-variables-aleatoire}

X : Longueur des sepales : variable quantitative

Hypothèse : H0 : muA = muB = muC H1 : au moins une moyenne est
différente des autres

Conditions d'applications X suit une loi normale -\textgreater{}
shapiro.test Homoscédasticité -\textgreater{} bartlett.test

-\textgreater{}\textgreater{} Si les conditions d'application sont
vérifiés alors on fait une ANOVA, sinon on fait un KRUSKAL-WALLIS

\subsection{Preparation des donnees:}\label{preparation-des-donnees}

\begin{Shaded}
\begin{Highlighting}[]
\KeywordTok{setwd}\NormalTok{(}\StringTok{"."}\NormalTok{)}

\KeywordTok{data}\NormalTok{(iris)}
\end{Highlighting}
\end{Shaded}

\begin{Shaded}
\begin{Highlighting}[]
\KeywordTok{summary}\NormalTok{(iris)}
\end{Highlighting}
\end{Shaded}

\begin{verbatim}
##   Sepal.Length    Sepal.Width     Petal.Length    Petal.Width   
##  Min.   :4.300   Min.   :2.000   Min.   :1.000   Min.   :0.100  
##  1st Qu.:5.100   1st Qu.:2.800   1st Qu.:1.600   1st Qu.:0.300  
##  Median :5.800   Median :3.000   Median :4.350   Median :1.300  
##  Mean   :5.843   Mean   :3.057   Mean   :3.758   Mean   :1.199  
##  3rd Qu.:6.400   3rd Qu.:3.300   3rd Qu.:5.100   3rd Qu.:1.800  
##  Max.   :7.900   Max.   :4.400   Max.   :6.900   Max.   :2.500  
##        Species  
##  setosa    :50  
##  versicolor:50  
##  virginica :50  
##                 
##                 
## 
\end{verbatim}

\subsection{Verification des conditions
d'application:}\label{verification-des-conditions-dapplication}

\subsubsection{Visualisation graphique des variables aleatoires pour les
3
especes}\label{visualisation-graphique-des-variables-aleatoires-pour-les-3-especes}

\begin{Shaded}
\begin{Highlighting}[]
\KeywordTok{par}\NormalTok{(}\DataTypeTok{mfrow =} \KeywordTok{c}\NormalTok{(}\DecValTok{2}\NormalTok{,}\DecValTok{2}\NormalTok{))}
\NormalTok{box1 =}\StringTok{ }\KeywordTok{boxplot}\NormalTok{ (Sepal.Length}\OperatorTok{~}\NormalTok{Species, }\DataTypeTok{data =}\NormalTok{ iris)}
\NormalTok{box2 =}\StringTok{ }\KeywordTok{boxplot}\NormalTok{ (Sepal.Width}\OperatorTok{~}\NormalTok{Species, }\DataTypeTok{data =}\NormalTok{ iris)}
\NormalTok{box3 =}\StringTok{ }\KeywordTok{boxplot}\NormalTok{ (Petal.Length}\OperatorTok{~}\NormalTok{Species, }\DataTypeTok{data =}\NormalTok{ iris)}
\NormalTok{box4 =}\StringTok{ }\KeywordTok{boxplot}\NormalTok{ (Petal.Width}\OperatorTok{~}\NormalTok{Species, }\DataTypeTok{data =}\NormalTok{ iris)}
\end{Highlighting}
\end{Shaded}

\includegraphics{Iris2_files/figure-latex/unnamed-chunk-3-1.pdf}

\begin{Shaded}
\begin{Highlighting}[]
\NormalTok{box1}
\end{Highlighting}
\end{Shaded}

\begin{verbatim}
## $stats
##      [,1] [,2] [,3]
## [1,]  4.3  4.9  5.6
## [2,]  4.8  5.6  6.2
## [3,]  5.0  5.9  6.5
## [4,]  5.2  6.3  6.9
## [5,]  5.8  7.0  7.9
## 
## $n
## [1] 50 50 50
## 
## $conf
##          [,1]     [,2]     [,3]
## [1,] 4.910622 5.743588 6.343588
## [2,] 5.089378 6.056412 6.656412
## 
## $out
## [1] 4.9
## 
## $group
## [1] 3
## 
## $names
## [1] "setosa"     "versicolor" "virginica"
\end{verbatim}

\begin{Shaded}
\begin{Highlighting}[]
\NormalTok{box2}
\end{Highlighting}
\end{Shaded}

\begin{verbatim}
## $stats
##      [,1] [,2] [,3]
## [1,]  2.9  2.0  2.2
## [2,]  3.2  2.5  2.8
## [3,]  3.4  2.8  3.0
## [4,]  3.7  3.0  3.2
## [5,]  4.4  3.4  3.8
## 
## $n
## [1] 50 50 50
## 
## $conf
##          [,1]     [,2]     [,3]
## [1,] 3.288277 2.688277 2.910622
## [2,] 3.511723 2.911723 3.089378
## 
## $out
## [1] 2.3
## 
## $group
## [1] 1
## 
## $names
## [1] "setosa"     "versicolor" "virginica"
\end{verbatim}

\begin{Shaded}
\begin{Highlighting}[]
\NormalTok{box3}
\end{Highlighting}
\end{Shaded}

\begin{verbatim}
## $stats
##      [,1] [,2] [,3]
## [1,]  1.1 3.30 4.50
## [2,]  1.4 4.00 5.10
## [3,]  1.5 4.35 5.55
## [4,]  1.6 4.60 5.90
## [5,]  1.9 5.10 6.90
## 
## $n
## [1] 50 50 50
## 
## $conf
##          [,1]     [,2]     [,3]
## [1,] 1.455311 4.215933 5.371243
## [2,] 1.544689 4.484067 5.728757
## 
## $out
## [1] 1 3
## 
## $group
## [1] 1 2
## 
## $names
## [1] "setosa"     "versicolor" "virginica"
\end{verbatim}

\begin{Shaded}
\begin{Highlighting}[]
\NormalTok{box4}
\end{Highlighting}
\end{Shaded}

\begin{verbatim}
## $stats
##      [,1] [,2] [,3]
## [1,]  0.1  1.0  1.4
## [2,]  0.2  1.2  1.8
## [3,]  0.2  1.3  2.0
## [4,]  0.3  1.5  2.3
## [5,]  0.4  1.8  2.5
## 
## $n
## [1] 50 50 50
## 
## $conf
##           [,1]     [,2]     [,3]
## [1,] 0.1776554 1.232966 1.888277
## [2,] 0.2223446 1.367034 2.111723
## 
## $out
## [1] 0.5 0.6
## 
## $group
## [1] 1 1
## 
## $names
## [1] "setosa"     "versicolor" "virginica"
\end{verbatim}

-\textgreater{}\textgreater{} On remarque qu'il y a en moyenne plus de
fleurs d'espece ``virginica'' qui possèdent de longues sepales. les
individus de l'espece ``setosa'' possèdent en moyenne des sepales plus
petites.

\subsubsection{Visualisation graphique de la normalité de distribution
des variables
aleatoires}\label{visualisation-graphique-de-la-normalite-de-distribution-des-variables-aleatoires}

\begin{Shaded}
\begin{Highlighting}[]
\KeywordTok{par}\NormalTok{(}\DataTypeTok{mfrow =} \KeywordTok{c}\NormalTok{(}\DecValTok{2}\NormalTok{,}\DecValTok{2}\NormalTok{))}

\CommentTok{#by(iris$Sepal.Length, iris$Species, hist)}
\KeywordTok{hist}\NormalTok{(iris}\OperatorTok{$}\NormalTok{Sepal.Length, }\DataTypeTok{main=} \StringTok{"Hist longueur des sepales"}\NormalTok{, }\DataTypeTok{xlab =} \StringTok{"longueur des sepales"}\NormalTok{, }\DataTypeTok{freq =}\NormalTok{ T)}
\KeywordTok{hist}\NormalTok{(iris}\OperatorTok{$}\NormalTok{Sepal.Width, }\DataTypeTok{main=} \StringTok{"Hist largeur des sepales"}\NormalTok{, }\DataTypeTok{xlab =} \StringTok{"largeur des sepales"}\NormalTok{, }\DataTypeTok{freq =}\NormalTok{ T)}
\KeywordTok{hist}\NormalTok{(iris}\OperatorTok{$}\NormalTok{Petal.Length, }\DataTypeTok{main=} \StringTok{"Hist longueur des petales"}\NormalTok{, }\DataTypeTok{xlab =} \StringTok{"longueur des petales"}\NormalTok{, }\DataTypeTok{freq =}\NormalTok{ T)}
\KeywordTok{hist}\NormalTok{(iris}\OperatorTok{$}\NormalTok{Petal.Width, }\DataTypeTok{main=} \StringTok{"Hist largeur des petales"}\NormalTok{, }\DataTypeTok{xlab =} \StringTok{"largeur des petales"}\NormalTok{, }\DataTypeTok{freq =}\NormalTok{ T)}
\end{Highlighting}
\end{Shaded}

\begin{Shaded}
\begin{Highlighting}[]
\KeywordTok{par}\NormalTok{ (}\DataTypeTok{mfrow =} \KeywordTok{c}\NormalTok{(}\DecValTok{1}\NormalTok{,}\DecValTok{3}\NormalTok{))}

\KeywordTok{by}\NormalTok{(iris}\OperatorTok{$}\NormalTok{Sepal.Length, iris}\OperatorTok{$}\NormalTok{Species, hist, }\DataTypeTok{xlab =} \StringTok{"longueur Sepale"}\NormalTok{, }\DataTypeTok{main =} \StringTok{"Hist longueur Sepale"}\NormalTok{)}
\end{Highlighting}
\end{Shaded}

\includegraphics{Iris2_files/figure-latex/unnamed-chunk-5-1.pdf}

\begin{verbatim}
## iris$Species: setosa
## $breaks
## [1] 4.2 4.4 4.6 4.8 5.0 5.2 5.4 5.6 5.8
## 
## $counts
## [1]  4  5  7 12 11  6  2  3
## 
## $density
## [1] 0.4 0.5 0.7 1.2 1.1 0.6 0.2 0.3
## 
## $mids
## [1] 4.3 4.5 4.7 4.9 5.1 5.3 5.5 5.7
## 
## $xname
## [1] "dd[x, ]"
## 
## $equidist
## [1] TRUE
## 
## attr(,"class")
## [1] "histogram"
## -------------------------------------------------------- 
## iris$Species: versicolor
## $breaks
## [1] 4.5 5.0 5.5 6.0 6.5 7.0
## 
## $counts
## [1]  3  8 19 12  8
## 
## $density
## [1] 0.12 0.32 0.76 0.48 0.32
## 
## $mids
## [1] 4.75 5.25 5.75 6.25 6.75
## 
## $xname
## [1] "dd[x, ]"
## 
## $equidist
## [1] TRUE
## 
## attr(,"class")
## [1] "histogram"
## -------------------------------------------------------- 
## iris$Species: virginica
## $breaks
## [1] 4.5 5.0 5.5 6.0 6.5 7.0 7.5 8.0
## 
## $counts
## [1]  1  0  8 19 10  6  6
## 
## $density
## [1] 0.04 0.00 0.32 0.76 0.40 0.24 0.24
## 
## $mids
## [1] 4.75 5.25 5.75 6.25 6.75 7.25 7.75
## 
## $xname
## [1] "dd[x, ]"
## 
## $equidist
## [1] TRUE
## 
## attr(,"class")
## [1] "histogram"
\end{verbatim}

\begin{Shaded}
\begin{Highlighting}[]
\KeywordTok{by}\NormalTok{(iris}\OperatorTok{$}\NormalTok{Sepal.Width, iris}\OperatorTok{$}\NormalTok{Species, hist, }\DataTypeTok{xlab =} \StringTok{"largeur Sepale"}\NormalTok{, }\DataTypeTok{main =} \StringTok{"Hist largeur Sepale"}\NormalTok{)}
\end{Highlighting}
\end{Shaded}

\includegraphics{Iris2_files/figure-latex/unnamed-chunk-5-2.pdf}

\begin{verbatim}
## iris$Species: setosa
## $breaks
## [1] 2.0 2.5 3.0 3.5 4.0 4.5
## 
## $counts
## [1]  1  7 26 13  3
## 
## $density
## [1] 0.04 0.28 1.04 0.52 0.12
## 
## $mids
## [1] 2.25 2.75 3.25 3.75 4.25
## 
## $xname
## [1] "dd[x, ]"
## 
## $equidist
## [1] TRUE
## 
## attr(,"class")
## [1] "histogram"
## -------------------------------------------------------- 
## iris$Species: versicolor
## $breaks
## [1] 2.0 2.2 2.4 2.6 2.8 3.0 3.2 3.4
## 
## $counts
## [1]  3  6  7 11 15  6  2
## 
## $density
## [1] 0.3 0.6 0.7 1.1 1.5 0.6 0.2
## 
## $mids
## [1] 2.1 2.3 2.5 2.7 2.9 3.1 3.3
## 
## $xname
## [1] "dd[x, ]"
## 
## $equidist
## [1] TRUE
## 
## attr(,"class")
## [1] "histogram"
## -------------------------------------------------------- 
## iris$Species: virginica
## $breaks
## [1] 2.2 2.4 2.6 2.8 3.0 3.2 3.4 3.6 3.8
## 
## $counts
## [1]  1  6 12 14  9  5  1  2
## 
## $density
## [1] 0.1 0.6 1.2 1.4 0.9 0.5 0.1 0.2
## 
## $mids
## [1] 2.3 2.5 2.7 2.9 3.1 3.3 3.5 3.7
## 
## $xname
## [1] "dd[x, ]"
## 
## $equidist
## [1] TRUE
## 
## attr(,"class")
## [1] "histogram"
\end{verbatim}

\begin{Shaded}
\begin{Highlighting}[]
\KeywordTok{by}\NormalTok{(iris}\OperatorTok{$}\NormalTok{Petal.Length, iris}\OperatorTok{$}\NormalTok{Species, hist, }\DataTypeTok{xlab =} \StringTok{"longueur Petale"}\NormalTok{, }\DataTypeTok{main =} \StringTok{"Hist longueur Sepale"}\NormalTok{)}
\end{Highlighting}
\end{Shaded}

\includegraphics{Iris2_files/figure-latex/unnamed-chunk-5-3.pdf}

\begin{verbatim}
## iris$Species: setosa
## $breaks
##  [1] 1.0 1.1 1.2 1.3 1.4 1.5 1.6 1.7 1.8 1.9
## 
## $counts
## [1]  2  2  7 13 13  7  4  0  2
## 
## $density
## [1] 0.4 0.4 1.4 2.6 2.6 1.4 0.8 0.0 0.4
## 
## $mids
## [1] 1.05 1.15 1.25 1.35 1.45 1.55 1.65 1.75 1.85
## 
## $xname
## [1] "dd[x, ]"
## 
## $equidist
## [1] TRUE
## 
## attr(,"class")
## [1] "histogram"
## -------------------------------------------------------- 
## iris$Species: versicolor
## $breaks
## [1] 3.0 3.5 4.0 4.5 5.0 5.5
## 
## $counts
## [1]  5 11 20 13  1
## 
## $density
## [1] 0.20 0.44 0.80 0.52 0.04
## 
## $mids
## [1] 3.25 3.75 4.25 4.75 5.25
## 
## $xname
## [1] "dd[x, ]"
## 
## $equidist
## [1] TRUE
## 
## attr(,"class")
## [1] "histogram"
## -------------------------------------------------------- 
## iris$Species: virginica
## $breaks
## [1] 4.5 5.0 5.5 6.0 6.5 7.0
## 
## $counts
## [1]  9 16 16  5  4
## 
## $density
## [1] 0.36 0.64 0.64 0.20 0.16
## 
## $mids
## [1] 4.75 5.25 5.75 6.25 6.75
## 
## $xname
## [1] "dd[x, ]"
## 
## $equidist
## [1] TRUE
## 
## attr(,"class")
## [1] "histogram"
\end{verbatim}

\begin{Shaded}
\begin{Highlighting}[]
\KeywordTok{by}\NormalTok{(iris}\OperatorTok{$}\NormalTok{Petal.Width, iris}\OperatorTok{$}\NormalTok{Species, hist, }\DataTypeTok{xlab =} \StringTok{"largeur Petale"}\NormalTok{, }\DataTypeTok{main =} \StringTok{"Hist largeur Petale"}\NormalTok{)}
\end{Highlighting}
\end{Shaded}

\includegraphics{Iris2_files/figure-latex/unnamed-chunk-5-4.pdf}

\begin{verbatim}
## iris$Species: setosa
## $breaks
## [1] 0.1 0.2 0.3 0.4 0.5 0.6
## 
## $counts
## [1] 34  7  7  1  1
## 
## $density
## [1] 6.8 1.4 1.4 0.2 0.2
## 
## $mids
## [1] 0.15 0.25 0.35 0.45 0.55
## 
## $xname
## [1] "dd[x, ]"
## 
## $equidist
## [1] TRUE
## 
## attr(,"class")
## [1] "histogram"
## -------------------------------------------------------- 
## iris$Species: versicolor
## $breaks
## [1] 1.0 1.1 1.2 1.3 1.4 1.5 1.6 1.7 1.8
## 
## $counts
## [1] 10  5 13  7 10  3  1  1
## 
## $density
## [1] 2.0 1.0 2.6 1.4 2.0 0.6 0.2 0.2
## 
## $mids
## [1] 1.05 1.15 1.25 1.35 1.45 1.55 1.65 1.75
## 
## $xname
## [1] "dd[x, ]"
## 
## $equidist
## [1] TRUE
## 
## attr(,"class")
## [1] "histogram"
## -------------------------------------------------------- 
## iris$Species: virginica
## $breaks
## [1] 1.4 1.6 1.8 2.0 2.2 2.4 2.6
## 
## $counts
## [1]  4 12 11  9 11  3
## 
## $density
## [1] 0.4 1.2 1.1 0.9 1.1 0.3
## 
## $mids
## [1] 1.5 1.7 1.9 2.1 2.3 2.5
## 
## $xname
## [1] "dd[x, ]"
## 
## $equidist
## [1] TRUE
## 
## attr(,"class")
## [1] "histogram"
\end{verbatim}

\subsubsection{Verification de la normalite des variables en fonction
des
especes}\label{verification-de-la-normalite-des-variables-en-fonction-des-especes}

\begin{Shaded}
\begin{Highlighting}[]
\KeywordTok{by}\NormalTok{(iris}\OperatorTok{$}\NormalTok{Sepal.Length, iris}\OperatorTok{$}\NormalTok{Species, shapiro.test)}
\end{Highlighting}
\end{Shaded}

\begin{verbatim}
## iris$Species: setosa
## 
##  Shapiro-Wilk normality test
## 
## data:  dd[x, ]
## W = 0.9777, p-value = 0.4595
## 
## -------------------------------------------------------- 
## iris$Species: versicolor
## 
##  Shapiro-Wilk normality test
## 
## data:  dd[x, ]
## W = 0.97784, p-value = 0.4647
## 
## -------------------------------------------------------- 
## iris$Species: virginica
## 
##  Shapiro-Wilk normality test
## 
## data:  dd[x, ]
## W = 0.97118, p-value = 0.2583
\end{verbatim}

\begin{Shaded}
\begin{Highlighting}[]
\KeywordTok{by}\NormalTok{(iris}\OperatorTok{$}\NormalTok{Sepal.Width, iris}\OperatorTok{$}\NormalTok{Species, shapiro.test)}
\end{Highlighting}
\end{Shaded}

\begin{verbatim}
## iris$Species: setosa
## 
##  Shapiro-Wilk normality test
## 
## data:  dd[x, ]
## W = 0.97172, p-value = 0.2715
## 
## -------------------------------------------------------- 
## iris$Species: versicolor
## 
##  Shapiro-Wilk normality test
## 
## data:  dd[x, ]
## W = 0.97413, p-value = 0.338
## 
## -------------------------------------------------------- 
## iris$Species: virginica
## 
##  Shapiro-Wilk normality test
## 
## data:  dd[x, ]
## W = 0.96739, p-value = 0.1809
\end{verbatim}

\begin{Shaded}
\begin{Highlighting}[]
\KeywordTok{by}\NormalTok{(iris}\OperatorTok{$}\NormalTok{Petal.Length, iris}\OperatorTok{$}\NormalTok{Species, shapiro.test)}
\end{Highlighting}
\end{Shaded}

\begin{verbatim}
## iris$Species: setosa
## 
##  Shapiro-Wilk normality test
## 
## data:  dd[x, ]
## W = 0.95498, p-value = 0.05481
## 
## -------------------------------------------------------- 
## iris$Species: versicolor
## 
##  Shapiro-Wilk normality test
## 
## data:  dd[x, ]
## W = 0.966, p-value = 0.1585
## 
## -------------------------------------------------------- 
## iris$Species: virginica
## 
##  Shapiro-Wilk normality test
## 
## data:  dd[x, ]
## W = 0.96219, p-value = 0.1098
\end{verbatim}

\begin{Shaded}
\begin{Highlighting}[]
\KeywordTok{by}\NormalTok{(iris}\OperatorTok{$}\NormalTok{Petal.Width, iris}\OperatorTok{$}\NormalTok{Species, shapiro.test)}
\end{Highlighting}
\end{Shaded}

\begin{verbatim}
## iris$Species: setosa
## 
##  Shapiro-Wilk normality test
## 
## data:  dd[x, ]
## W = 0.79976, p-value = 8.659e-07
## 
## -------------------------------------------------------- 
## iris$Species: versicolor
## 
##  Shapiro-Wilk normality test
## 
## data:  dd[x, ]
## W = 0.94763, p-value = 0.02728
## 
## -------------------------------------------------------- 
## iris$Species: virginica
## 
##  Shapiro-Wilk normality test
## 
## data:  dd[x, ]
## W = 0.95977, p-value = 0.08695
\end{verbatim}

-\textgreater{}\textgreater{} D'apres les histogrammes et les resultats
du shapiro.test, avec un risque alpha = 0.05, seules les variables
largeur de la petale pour l'espece setosa et versicolor ne suivent pas
une loi normale. Toutes les autres varaibles suivent une loi normale.

\subsubsection{Verification de l'homoscedasticite de la variable largeur
des
sepales:}\label{verification-de-lhomoscedasticite-de-la-variable-largeur-des-sepales}

H0: sigma²A = sigma²B = sigma²C H1: une des variance est differente

\begin{Shaded}
\begin{Highlighting}[]
\KeywordTok{bartlett.test}\NormalTok{(Sepal.Length}\OperatorTok{~}\NormalTok{Species, }\DataTypeTok{data =}\NormalTok{ iris)}
\end{Highlighting}
\end{Shaded}

\begin{verbatim}
## 
##  Bartlett test of homogeneity of variances
## 
## data:  Sepal.Length by Species
## Bartlett's K-squared = 16.006, df = 2, p-value = 0.0003345
\end{verbatim}

\begin{Shaded}
\begin{Highlighting}[]
\KeywordTok{bartlett.test}\NormalTok{(Sepal.Width}\OperatorTok{~}\NormalTok{Species, }\DataTypeTok{data =}\NormalTok{ iris)}
\end{Highlighting}
\end{Shaded}

\begin{verbatim}
## 
##  Bartlett test of homogeneity of variances
## 
## data:  Sepal.Width by Species
## Bartlett's K-squared = 2.0911, df = 2, p-value = 0.3515
\end{verbatim}

\begin{Shaded}
\begin{Highlighting}[]
\KeywordTok{bartlett.test}\NormalTok{(Petal.Length}\OperatorTok{~}\NormalTok{Species, }\DataTypeTok{data =}\NormalTok{ iris)}
\end{Highlighting}
\end{Shaded}

\begin{verbatim}
## 
##  Bartlett test of homogeneity of variances
## 
## data:  Petal.Length by Species
## Bartlett's K-squared = 55.423, df = 2, p-value = 9.229e-13
\end{verbatim}

-\textgreater{}\textgreater{} D'apres les resultats du test, avec un
risque alpha, seules les variances de la variable largeur de sepales
sont identiques.


\end{document}
